\documentclass[10pt]{jarticle}
\usepackage{mgsh}
\usepackage{enumerate}
\usepackage{ascmac} % screen

\usepackage{geometry} % 余白など指定
\geometry{left=15mm,right=15mm,top=20mm,bottom=20mm} % 余白指定

\SetupMagicSheet{question/name=Question,solution/name=Answer} % 大問(設問)のラベルが左記に変更される

\IsVisible{true}  % true=表示,コメントアウト(コマンド無)/false=非表示

\begin{document}

\center{\Large{Magic Sheetの簡易説明}}
\begin{enumerate}
  \item 大問の設置
    \begin{itemize}
      \item 書式:\textbackslash begin\{question\}〜\textbackslash end\{question\}
      \item 例:  \textbackslash begin\{question\}大問設置.個々の問題を記述 \textbackslash end\{question\} 
    \item 結果:\begin{screen}\begin{question}大問設置.個々の問題を記述 \end{question}\end{screen}
      \item 補足:大問のラベルを変更したい場合は\textbackslash SetupMgshSheetsコマンドを通して変更できます 
    \end{itemize}                                                                               
  \item 大問の解答の設置
    \begin{itemize}
      \item 書式:\textbackslash begin\{solution\}〜\textbackslash end\{solution\}
      \item 例:  \textbackslash begin\{solution\}大問解答の設置. \textbackslash end\{solution\} 
    \item 結果:\begin{screen}\begin{solution} 大問解答の設置. \end{solution} \end{screen}
      \item 補足:大問の解答のラベルを変更したい場合は\textbackslash SetupMgshSheetsコマンドを通して変更できます 
    \end{itemize}
  \item 穴抜き問題の例:
    \begin{itemize}
      \item 書式:\textbackslash mgshBox\{解答\}\{ラベル名\}\{解答への捕捉(表示)\}
        \begin{itemize}
          \item 解答:穴抜き問題の解答
          \item ラベル名:設問を識別するための識別子 
        \item 解答への捕捉(表示):欄内に補記したいメッセージ文
      \end{itemize}
    \item 例:\textbackslash mgshBox\{answer\}\{testLabel\}\{(アルファベットで解答する事)\}
    \item 結果:\mgshBox{answer}{testLabel}{(アルファベットで解答する事)}
  \end{itemize}
  \item 設問番号の自動付与
    \begin{itemize}
      \item 書式:\textbackslash mgshNo\{ラベル名\}
        \begin{itemize}
          \item ラベル名:設問を識別するための識別子
        \end{itemize}
      \item 例: \textbackslash mgshNo\{testLabel\}
      \item 結果: \mgshNo{testLabel}
    \end{itemize}
  \item 設問番号の参照:
    \begin{itemize}
      \item 書式:\textbackslash mgshRef\{ラベル名\}
        \begin{itemize}
          \item ラベル名:設問を識別するための識別子 
        \end{itemize}
      \item 例:  \textbackslash mgshRef\{testLabel\} 
      \item 結果:\mgshRef{testLabel} ※上記の\textbackslash mgshNo\{testLabel\}を参照している
    \end{itemize}
  \item 解答欄の生成,欄のサイズ指定可
    \begin{itemize}
      \item 書式: \textbackslash mgshBoxShape\{width\}\{height\}\{解答\}\{ラベル名\}
      \item 例:  \textbackslash mgshBoxShape\{100\}\{20\}\{答え\}\{testBoxShape\}
      \item 結果:\mgshBoxShape{100}{20}{答え}{testBoxShape}
    \end{itemize}
\end{enumerate}

\end{document}                                                  
